\documentclass{article}
% This is a LaTeX file.  It is a text file that is compiled
% This is a LaTeX file.  It is a text file that is compiled
% by a program called LaTeX into a pretty PDF file.  
% If you're viewing this file on Overleaf, 
% you'll see that PDF in the window to the right.
%
% The LaTeX macro language is complicated, so we have inserted
% lots of documenting comments into the file.  Comments start
% with `%' and continue to the end of the line.  In Overleaf's
% window, they are colored blue.
%
% Comments prefixed with `Student:' are relevant to students.
% Skip anything else you don't understand, or ask me.
%
% set font encoding for PDFLaTeX or XeLaTeX
\usepackage{ifxetex}
\ifxetex
  \usepackage{fontspec}
\else
  \usepackage[T1]{fontenc}
  \usepackage[utf8]{inputenc}
  \usepackage{lmodern}
\fi

% Student: These lines describe some document metadata.
\title{Problem Set 1}
\author{%
% Student: change the next line to your name!
    Name
\\  MATH-UA 120 Discrete Mathematics
}
\date{due February 2, 2024}


\usepackage[headings=runin-fixed-nr]{exsheets}
% These make enumerates within questions start at the second ("(a)") level, rather than the first ("1.") level.
\makeatletter
    \newcommand{\stepenumdepth}{\advance\@enumdepth\@ne}
\makeatother
\SetupExSheets{
    question/pre-body-hook=\stepenumdepth,
    solution/pre-body-hook=\stepenumdepth,
}
\DeclareInstance{exsheets-heading}{runin-nn-np}{default}{
    runin = true,
    title-post-code = .\space,
    join = {
        main[r,vc]title[l,vc](0pt,0pt);
    }
}
\newif\ifshowsolutions
% Student: replace `false' with `true' to typeset your solutions.
% Otherwise they are ignored!
\showsolutionstrue
\ifshowsolutions
    \SetupExSheets{
        question/pre-hook=\itshape,
        solution/headings=runin-nn-np,
        solution/print=true,
        solution/name=Answer
    }%
    \makeatletter%
    \pretocmd{\@title}{Answers to }%
    \makeatother%
\else
    \SetupExSheets{solution/print=false}
\fi

% Bug workaround: http://tex.stackexchange.com/a/146536/1402
%\newenvironment{exercise}{}{}
\RenewQuSolPair{question}{solution}
%\let\answer\solution
%\let\endanswer\endsolution
\usepackage{manfnt}
\newcommand{\danger}{\marginpar[\hfill\dbend]{\dbend\hfill}}

\usepackage{amsmath, amsthm}
\usepackage{amsfonts}
\usepackage{enumerate}
\usepackage{siunitx}
\DeclareSIUnit\pound{lb}
\usepackage{hyperref}
\newtheorem*{theorem}{Theorem}
\newtheorem*{claim}{Claim}
\theoremstyle{definition}
\newtheorem*{definition}{Definition}
% We are creating a command "\xor".
\newcommand{\xor}{\underline{\lor}}
% This is the beginning of the part of the file that describes
% the text of the document.
% That's why it says `\begin{document}' below. :-)
\begin{document}
\maketitle



These are to be written up and turned in to Gradescope.\\



\ifshowsolutions
    \SetupExSheets{solution/print=true}
\else
    \danger
 \underline{ \LaTeX  Instructions:}  You can view the source (\texttt{.tex}) file to get some more examples of \LaTeX{} code.  I have commented in the source file in places where new \LaTeX{} constructions are used.
  
  Remember to change \verb|\showsolutionsfalse| to \verb|\showsolutionstrue|
    in the document's preamble 
    (between \verb|\documentclass{article}| and \verb|\begin{document}|)
\fi

\section*{Assigned Problems}


\begin{question}{4}
    % Notice the use of the enumerate environment
    % to make a numbered list.  Each item is marked
    % by the \item command.
    %
    % Also \emph = emphasize, usually in italics.
    Let the following statements be given. 
       \begin{definition}
          A triangle is \emph{scalene} if all of its sides have different lengths.
       \end{definition}
       \begin{theorem}
          A triangle is scalene if it is a right triangle that is not isosceles.
       \end{theorem}
    Suppose $\Delta ABC$ is a scalene triangle. 
    Which of the following conclusions are valid based only on the information given above? 
    Why or why not?
    \begin{enumerate}
        \item All of the sides of $\Delta ABC$ have different lengths.
        \item $\Delta ABC$ is a right triangle that is not isosceles.
    \end{enumerate}
\end{question}
% Student: put your answer between the next two lines.
\begin{solution}
    \begin{enumerate}
        \item This is valid based on the definition. Note that definitions are "if and only if" statements.
        \item This is not valid based on the theorem, since we are given the conclusion, 
            we cannot say anything about the hypothesis.
    \end{enumerate}
    
{\color{red} Rubric:
\begin{itemize}
\item 2P per part:
\item 1P for answer
\item 1P for reasoning
\end{itemize}}
\end{solution}


\begin{question}{6}
   Without changing their meanings, convert each of the following sentences into a sentence of the form ``For all ... $x$, if $x$ ... , then .''
    \begin{enumerate}
        \item Every prime greater than 2 is odd.
        \item Three consecutive odd integers greater than 3 cannot all be prime.
        \item An integer is divisible by 8 only if it is divisible by 4.
    \end{enumerate}
\end{question}
% Student: put your answer between the next two lines.
\begin{solution}
    \begin{enumerate}
        \item For all integers $x$, if $x>2$ and prime, then $x$ is odd. 
        \item For all integers $x$, if $x$, $x+2$, and $x+4$ are all greater than 3, then they cannot all be prime. 
        \item For all integers $x$, if $x$ is divisible by 8, then $x$ is divisible by 4.
    \end{enumerate}

{\color{red} Rubric:
\begin{itemize}
\item 2P for statement per part:
\item 1P for correct order of "if, then"
\item 1P for correct object in "For all ... x", such as "integers", etc
\item There may be alternative correct solutions. For example, For all primes $p$, if $p>2$, then $p$ is odd.
\end{itemize}}
\end{solution}


\begin{question}{10}
   The following claim and its proof are poorly written. They are both missing some crucial information. 
   Explain why both the claim and its proof are poorly written. Then revise both the claim and proof so that any student in this course will understand it. 
      \begin{claim}
       If $x^2\neq 0$, then $x^2>0$.
      \end{claim}
      \begin{proof}
       If $x>0$, then $x=xx>0$. If $x<0$, then $-x>0$, so $(-x)(-x)>0$, i.e., $x^2>0$.
      \end{proof}
\end{question}
% Student: put your answer between the next two lines.
\begin{solution}
    The claim has no information about the quantity $x$. A clearer statement would be:
        \begin{claim}
           For all real numbers $x$, if $x^2\neq 0$, then $x^2>0$.
        \end{claim}
    In the proof, the assumption $x^2\neq 0$ is missing. 
    From there, it looks like the original proof is broken into cases as a result of the hypothesis. 
    Here is a clearer version of the proof:
        \begin{proof}
           Let $x$ be a real number such that $x^2\neq 0$. Then $x\neq 0$, and we have two cases:
            \begin{enumerate}[i.]
                 \item if $x>0$, then $x\cdot x=x^2>0$.
                 \item if $x<0$, then $-x>0$, so $(-x)(-x)>0$, that is, $x^2>0$.
            \end{enumerate}
           Therefore, in both cases, $x^2>0$.
        \end{proof}

{\color{red} Rubric:
\begin{itemize}
\item Claim: 2P for fix + 2P for reasoning
\item Proof: 4P for fix + 2P for reasoning
\item There may be alternative correct solutions. For example, $x$ doesn't have to be reals to make this a true statement.
\end{itemize}}
\end{solution}

\begin{question}{6}
    For each of the following parts, explain why the given argument is not a valid proof.
    \begin{enumerate}
        \item \begin{theorem} For all primes $p$, the integer $2^p-2$ is divisible by $p$.
        		\end{theorem}
		\begin{proof}[Wrong Proof] 
		\[
		2^2 - 2 = 2 \cdot 1, \quad  2^3 - 2 = 3 \cdot 2, \quad 2^5 - 2 = 5 \cdot 6, \quad 2^7 - 2 = 7 \cdot 18 \quad \text{etc.}
		\]
		\end{proof}
		
	\item \begin{theorem} For all $x, y$, and $z$, if $x+y <  x+z$, then $y < z$.
        		\end{theorem}
		\begin{proof}[Wrong Proof] 
		Suppose that $x+y <  x+z$. Take $x=0$. Then 
		\[
		y = 0 + y < 0 + z = z.
		\]
		\end{proof}
		
	\item \begin{theorem} $\sqrt{2} + \sqrt{6}  <  \sqrt{15}$.
        		\end{theorem}
		\begin{proof}[Wrong Proof] 
		\begin{align*}
		\sqrt{2} + \sqrt{6}  <  \sqrt{15} &\implies (\sqrt{2} + \sqrt{6})^2  <  15\\
		&\implies 8 + 2\sqrt{12} < 15\\
		&\implies 2\sqrt{12} < 7\\
		&\implies 48 < 49.
		\end{align*}
		\end{proof}
    \end{enumerate}
\end{question}
% Student: put your answer between the next two lines.
\begin{solution}
    \begin{enumerate}
      \item The theorem has been proven only for $p = 2, 3, 5, 7$, not all primes.
      \item Although, $y$ and $z$ are arbitrary, the theorem has been proven only for the special case of $x=0$, not any $x$.
      \item The goal is to prove our statement is true. Instead we already assumed it is true and showed another true statement. This gives us no information regarding whether the original statement is true or not.
    \end{enumerate}
    
{\color{red} Rubric:
\begin{itemize}
\item 2P per part
\end{itemize}}
\end{solution}


\begin{question}{15}
    Consider the following definitions.
	\begin{definition}
	 An integer $n$ is \textbf{alphic} if $n=4k+1$ for some integer $k$.
	\end{definition}
	\begin{definition}
	 An integer $n$ is \textbf{gammic} if $n=4k+3$ for some integer $k$.
	\end{definition}
        \begin{enumerate}
           \item Show that $-17$ is gammic.
           \item Give an example of an alphic integer and explain why it is alphic.
           \item Explain whether this statement is true: ``If $6$ is alphic, then $8$ is gammic.''
           \item Prove that if $x$ is alphic and $y$ is gammic, then $x+y$ is even.
        \end{enumerate}
\end{question}
% Student: put your answer between the next two lines.
\begin{solution}
    \begin{enumerate}
      \item We want to find an integer $k$ that satisfies $-17=4k+3$. Let $k=-5$, which is an integer. Note that $-17 = 4(-5) + 3$. Therefore, $-17$ is gammic.
      \item Consider the integer $1$. Note that $1 = 4(0) +1$ where $0$ is an integer. Therefore $1$ is alphic.
      \item The statement ``If $6$ is alphic, then $8$ is gammic.'' is vacuously true because the sufficient condition is false, since $6$ is not alphic.
      \item Let $x$ and $y$ be integers such that $x$ is alphic and $y$ is gammic. Then there exist integers $a$ and $b$ such that $x=4a+1$ and $y=4b+3$. Observe
      \begin{align*}
      x + y & = (4a+1) + (4b+3)\\
      & = 4a + 4b + 4\\
      & = 2 ( 2a + 2b + 2).
      \end{align*}
      Since $2a + 2b + 2\in \mathbb{Z}$, $x + y$ is even.
    \end{enumerate}
    
{\color{red} Rubric:
\begin{itemize}
\item 2P for part a
\item 2P for part b
\item 2P for part c
\item Part d: follow RVF 9P rubric
\end{itemize}}
\end{solution}

\begin{question}{9}
    Prove that an integer $n$ is odd if and only if $2n+2$ is divisible by 4.
\end{question}
% Student: put your answer between the next two lines.
\begin{solution}
    This statement can be written as 
	\begin{center}
		$n \in \mathbb{Z}$ is odd $\iff 4 \mid (2n+2)$.
	\end{center}
	Let $n\in\mathbb{Z}$.
\begin{itemize}	
    \item[($\Rightarrow$)] Assume first that $n$ is odd. Then we can write $n = 2k+1$ for some integer $k$. Then
	\[
	2 n + 2 = 2 (2k + 1) + 2 = 4 k + 2 + 2 = 4 k + 4 = 4 (k+1).
	\]
	Since $k+1 \in \mathbb{Z}$, it means by definition that $2 n + 2$ is divisible by 4.
	
    \item[($\Leftarrow$)] Assume now that $2n + 2$ is divisible by 4. 
    Then we can write $2n + 2 = 4k$ for some $k \in \mathbb{Z}$. Simplifying gives $n + 1 = 2k$, and thus
	\[
	n = 2 k - 1 = 2 (k - 1) + 1.
	\]
    Since $k - 1 \in \mathbb{Z}$, this means by definition that $n$ is odd.
    
    
\end{itemize}
{\color{red} Rubric:
\begin{itemize}
\item Follow RVF 9P rubric 
\end{itemize}}
\end{solution}


\end{document}
