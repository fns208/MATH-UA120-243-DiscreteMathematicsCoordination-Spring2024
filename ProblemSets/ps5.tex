\documentclass{article}
% This is a LaTeX file.  It is a text file that is compiled
% by a program called LaTeX into a pretty PDF file.  
% If you're viewing this file on Overleaf, 
% you'll see that PDF in the window to the right.
%
% The LaTeX macro language is complicated, so we have inserted
% lots of documenting comments into the file.  Comments start
% with `%' and continue to the end of the line.  In Overleaf's
% window, they are colored blue.
%
% Comments prefixed with `Student:' are relevant to students.
% Skip anything else you don't understand, or ask me.
%
% set font encoding for PDFLaTeX or XeLaTeX
\usepackage{ifxetex}
\ifxetex
  \usepackage{fontspec}
\else
  \usepackage[T1]{fontenc}
  \usepackage[utf8]{inputenc}
  \usepackage{lmodern}
\fi

% Student: These lines describe some document metadata.
\title{Problem Set 5}
\author{%
% Student: change the next line to your name!
    Name
\\  MATH-UA 120 Discrete Mathematics
}
\date{due March 15, 2024}


\usepackage[headings=runin-fixed-nr]{exsheets}
% These make enumerates within questions start at the second ("(a)") level, rather than the first ("1.") level.
\makeatletter
    \newcommand{\stepenumdepth}{\advance\@enumdepth\@ne}
\makeatother
\SetupExSheets{
    question/pre-body-hook=\stepenumdepth,
    solution/pre-body-hook=\stepenumdepth,
}
\DeclareInstance{exsheets-heading}{runin-nn-np}{default}{
    runin = true,
    title-post-code = .\space,
    join = {
        main[r,vc]title[l,vc](0pt,0pt);
    }
}
\newif\ifshowsolutions
% Student: replace `false' with `true' to typeset your solutions.
% Otherwise they are ignored!
\showsolutionsfalse
\ifshowsolutions
    \SetupExSheets{
        question/pre-hook=\itshape,
        solution/headings=runin-nn-np,
        solution/print=true,
        solution/name=Answer
    }%
    \makeatletter%
    \pretocmd{\@title}{Answers to }%
    \makeatother%
\else
    \SetupExSheets{solution/print=false}
\fi

% Bug workaround: http://tex.stackexchange.com/a/146536/1402
%\newenvironment{exercise}{}{}
\RenewQuSolPair{question}{solution}
%\let\answer\solution
%\let\endanswer\endsolution
\usepackage{manfnt}
\newcommand{\danger}{\marginpar[\hfill\dbend]{\dbend\hfill}}

% We are creating a command for some common commands.
\newcommand{\Z}{\mathbb{Z}}
\newcommand{\N}{\mathbb{N}}

% This package is for specifying graphics.  It's amazing.
% Manual at http://texdoc.net/texmf-dist/doc/generic/pgf/pgfmanual.pdf
\usepackage{tikz}

\usepackage{amsmath, amsthm, amssymb}
\usepackage{amsfonts}
\usepackage{siunitx}
\DeclareSIUnit\pound{lb}
\usepackage{hyperref}
\newtheorem*{theorem}{Theorem}
\newtheorem*{falseclaim}{False Claim}
\theoremstyle{definition}
\newtheorem*{definition}{Definition}
% This is the beginning of the part of the file that describes
% the text of the document.
% That's why it says `\begin{document}' below. :-)
\begin{document}
\maketitle



These are to be written up in \LaTeX{} and turned in to Gradescope.\\



\ifshowsolutions
    \SetupExSheets{solution/print=true}
\else
    \danger
 \underline{ \LaTeX  Instructions:}  You can view the source (\texttt{.tex}) file \href{https://bit.ly/43f8Zmb}{here} to get some more examples of \LaTeX{} code.  I have commented in the source file in places where new \LaTeX{} constructions are used.
  
  Remember to change \verb|\showsolutionsfalse| to \verb|\showsolutionstrue|
    in the document's preamble 
    (between \verb|\documentclass{article}| and \verb|\begin{document}|)
\fi

\section*{Assigned Problems}

\begin{question}{5}
    Prove the following statement by contrapositive: \\
    For all $n\in \mathbb{N}$, if $2^n<n!$, then $n>3$.
\end{question}
% Student: put your answer between the next two lines.
\begin{solution}
\end{solution}

\begin{question}{5}
    Prove the following statement by contrapositive: \\
    For all $a, b\in \mathbb{Z}$, if $a^2(b^2-2b)$ is odd, then $a$ and $b$ are odd.
\end{question}
% Student: put your answer between the next two lines.
\begin{solution}
\end{solution}


\begin{question}{5}
    Prove the following by contradiction:\\
    Let $A, B, C$ be sets. If $A\subseteq B$ and $B\cap C=\emptyset$, then $A\cap C=\emptyset$.
\end{question}
% Student: put your answer between the next two lines.
\begin{solution}
\end{solution}

\begin{question}{5}
    Prove the following statement by contradiction:\\
    Let $x, y\in \Z$. Then $x^2-4y-3\neq 0$.
\end{question}
% Student: put your answer between the next two lines.
\begin{solution}
\end{solution}

\begin{question}{6}
    Prove the following by smallest counterexample:\\
    Let $n\in \mathbb{N}$. If $n\geq 1$, then $4 \mid (5^n-1)$.
\end{question}
% Student: put your answer between the next two lines.
\begin{solution}
\end{solution}

\begin{question}{4}
    Find the logic error in the proof. Explain why the proof is not valid; meaning, why is the proof logically wrong (\textit{do not say because the claim is false}).
    \begin{falseclaim}
    Let $n\in \mathbb{N}$. $(*) \ 2n=0$.
    \end{falseclaim}
    
    \begin{proof}
    We prove the claim using proof by contradiction.
    
     For the sake of contradiction, assume there exists a natural number such that $(*)$ does not hold. By the Well-Ordering Principle, there exists a smallest $x\in \N$ such that $2x\neq 0$. Note $x\neq 0$. Then there exist  $i, j\in \N$ such that $i, j<x$ and $i+j=x$. Since $i, j <x$, $2i=0$ and $2j=0$. Observe
      \begin{align*}
      2x &= 2(i+j)\\
      & = 2i+2j\\
      & = 0.
      \end{align*}
      Contradiction! Therefore, $(*)$ for $n\in \N$.
    \end{proof}
\end{question}
% Student: put your answer between the next two lines.
\begin{solution}
\end{solution}

\begin{question}{10}
    Let $n\in \Z$. Use induction to prove that $\binom{2n}{n} < 2^{2n-2}$ for all $n\geq 5$. 
\end{question}
% Student: put your answer between the next two lines.
\begin{solution}
% Here is an example on how to label your bullets to follow the induction template.
%	\begin{description}
%	\item[Base Cases: ] 
%	\item[Inductive Hypothesis: ] 
%	\item[Inductive Step: ]
%	\end{description}
\end{solution}

\begin{question}{10}
    Let $n\in \Z$. Use induction to prove that $3 \mid (n^3+2n)$. \textit{Note: We want to prove the claim for all integers, not just natural numbers.}
\end{question}
% Student: put your answer between the next two lines.
\begin{solution}
\end{solution}








\end{document}
