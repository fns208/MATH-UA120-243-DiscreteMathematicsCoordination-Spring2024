\documentclass{article}
% This is a LaTeX file.  It is a text file that is compiled
% This is a LaTeX file.  It is a text file that is compiled
% by a program called LaTeX into a pretty PDF file.  
% If you're viewing this file on Overleaf, 
% you'll see that PDF in the window to the right.
%
% The LaTeX macro language is complicated, so we have inserted
% lots of documenting comments into the file.  Comments start
% with `%' and continue to the end of the line.  In Overleaf's
% window, they are colored blue.
%
% Comments prefixed with `Student:' are relevant to students.
% Skip anything else you don't understand, or ask me.
%
% set font encoding for PDFLaTeX or XeLaTeX
\usepackage{ifxetex}
\ifxetex
  \usepackage{fontspec}
\else
  \usepackage[T1]{fontenc}
  \usepackage[utf8]{inputenc}
  \usepackage{lmodern}
\fi

% Student: These lines describe some document metadata.
\title{Problem Set 1}
\author{%
% Student: change the next line to your name!
    Name
\\  MATH-UA 120 Discrete Mathematics
}
\date{due February 2, 2024}


\usepackage[headings=runin-fixed-nr]{exsheets}
% These make enumerates within questions start at the second ("(a)") level, rather than the first ("1.") level.
\makeatletter
    \newcommand{\stepenumdepth}{\advance\@enumdepth\@ne}
\makeatother
\SetupExSheets{
    question/pre-body-hook=\stepenumdepth,
    solution/pre-body-hook=\stepenumdepth,
}
\DeclareInstance{exsheets-heading}{runin-nn-np}{default}{
    runin = true,
    title-post-code = .\space,
    join = {
        main[r,vc]title[l,vc](0pt,0pt);
    }
}
\newif\ifshowsolutions
% Student: replace `false' with `true' to typeset your solutions.
% Otherwise they are ignored!
\showsolutionsfalse
\ifshowsolutions
    \SetupExSheets{
        question/pre-hook=\itshape,
        solution/headings=runin-nn-np,
        solution/print=true,
        solution/name=Answer
    }%
    \makeatletter%
    \pretocmd{\@title}{Answers to }%
    \makeatother%
\else
    \SetupExSheets{solution/print=false}
\fi

% Bug workaround: http://tex.stackexchange.com/a/146536/1402
%\newenvironment{exercise}{}{}
\RenewQuSolPair{question}{solution}
%\let\answer\solution
%\let\endanswer\endsolution
\usepackage{manfnt}
\newcommand{\danger}{\marginpar[\hfill\dbend]{\dbend\hfill}}

\usepackage{amsmath, amsthm}
\usepackage{amsfonts}
\usepackage{enumerate}
\usepackage{siunitx}
\DeclareSIUnit\pound{lb}
\usepackage{hyperref}
\newtheorem*{theorem}{Theorem}
\newtheorem*{claim}{Claim}
\theoremstyle{definition}
\newtheorem*{definition}{Definition}
% We are creating a command "\xor".
\newcommand{\xor}{\underline{\lor}}
% This is the beginning of the part of the file that describes
% the text of the document.
% That's why it says `\begin{document}' below. :-)
\begin{document}
\maketitle



These are to be written up and turned in to Gradescope.\\



\ifshowsolutions
    \SetupExSheets{solution/print=true}
\else
    \danger
 \underline{ \LaTeX  Instructions:}  You can view the source (\texttt{.tex}) file to get some more examples of \LaTeX{} code.  I have commented in the source file in places where new \LaTeX{} constructions are used.
  
  Remember to change \verb|\showsolutionsfalse| to \verb|\showsolutionstrue|
    in the document's preamble 
    (between \verb|\documentclass{article}| and \verb|\begin{document}|)
\fi

\section*{Assigned Problems}


\begin{question}{4}
    % Notice the use of the enumerate environment
    % to make a numbered list.  Each item is marked
    % by the \item command.
    %
    % Also \emph = emphasize, usually in italics.
    Let the following statements be given. 
       \begin{definition}
          A triangle is \emph{scalene} if all of its sides have different lengths.
       \end{definition}
       \begin{theorem}
          A triangle is scalene if it is a right triangle that is not isosceles.
       \end{theorem}
    Suppose $\Delta ABC$ is a scalene triangle. 
    Which of the following conclusions are valid based only on the information given above? 
    Why or why not?
    \begin{enumerate}
        \item All of the sides of $\Delta ABC$ have different lengths.
        \item $\Delta ABC$ is a right triangle that is not isosceles.
    \end{enumerate}
\end{question}
% Student: put your answer between the next two lines.
\begin{solution}
\end{solution}


\begin{question}{6}
   Without changing their meanings, convert each of the following sentences into a sentence of the form ``For all ... $x$, if $x$ ... , then .''
    \begin{enumerate}
        \item Every prime greater than 2 is odd.
        \item Three consecutive odd integers greater than 3 cannot all be prime.
        \item An integer is divisible by 8 only if it is divisible by 4.
    \end{enumerate}
\end{question}
% Student: put your answer between the next two lines.
\begin{solution}
\end{solution}


\begin{question}{10}
   The following claim and its proof are poorly written. They are both missing some crucial information. 
   Explain why both the claim and its proof are poorly written. Then revise both the claim and proof so that any student in this course will understand it. 
      \begin{claim}
       If $x^2\neq 0$, then $x^2>0$.
      \end{claim}
      \begin{proof}
       If $x>0$, then $x^2=xx>0$. If $x<0$, then $-x>0$, so $(-x)(-x)>0$, i.e., $x^2>0$.
      \end{proof}
\end{question}
% Student: put your answer between the next two lines.
\begin{solution}
\end{solution}

\begin{question}{6}
    For each of the following parts, explain why the given argument is not a valid proof.
    \begin{enumerate}
        \item \begin{theorem} For all primes $p$, the integer $2^p-2$ is divisible by $p$.
        		\end{theorem}
		\begin{proof}[Wrong Proof] 
		\[
		2^2 - 2 = 2 \cdot 1, \quad  2^3 - 2 = 3 \cdot 2, \quad 2^5 - 2 = 5 \cdot 6, \quad 2^7 - 2 = 7 \cdot 18 \quad \text{etc.}
		\]
		\end{proof}
		
	\item \begin{theorem} For all $x, y$, and $z$, if $x+y <  x+z$, then $y < z$.
        		\end{theorem}
		\begin{proof}[Wrong Proof] 
		Suppose that $x+y <  x+z$. Take $x=0$. Then 
		\[
		y = 0 + y < 0 + z = z.
		\]
		\end{proof}
		
	\item \begin{theorem} $\sqrt{2} + \sqrt{6}  <  \sqrt{15}$.
        		\end{theorem}
		\begin{proof}[Wrong Proof] 
		\begin{align*}
		\sqrt{2} + \sqrt{6}  <  \sqrt{15} &\implies (\sqrt{2} + \sqrt{6})^2  <  15\\
		&\implies 8 + 2\sqrt{12} < 15\\
		&\implies 2\sqrt{12} < 7\\
		&\implies 48 < 49.
		\end{align*}
		\end{proof}
    \end{enumerate}
\end{question}
% Student: put your answer between the next two lines.
\begin{solution}
\end{solution}


\begin{question}{15}
    Consider the following definitions.
	\begin{definition}
	 An integer $n$ is \textbf{alphic} if $n=4k+1$ for some integer $k$.
	\end{definition}
	\begin{definition}
	 An integer $n$ is \textbf{gammic} if $n=4k+3$ for some integer $k$.
	\end{definition}
        \begin{enumerate}
           \item Show that $-17$ is gammic.
           \item Give an example of an alphic integer and explain why it is alphic.
           \item Explain whether this statement is true: ``If $6$ is alphic, then $8$ is gammic.''
           \item Prove that if $x$ is alphic and $y$ is gammic, then $x+y$ is even.
        \end{enumerate}
\end{question}
% Student: put your answer between the next two lines.
\begin{solution}
\end{solution}

\begin{question}{9}
    Prove that an integer $n$ is odd if and only if $2n+2$ is divisible by 4.
\end{question}
% Student: put your answer between the next two lines.
\begin{solution}
\end{solution}


\end{document}
