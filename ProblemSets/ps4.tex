\documentclass{article}
% This is a LaTeX file.  It is a text file that is compiled
% by a program called LaTeX into a pretty PDF file.  
% If you're viewing this file on Overleaf, 
% you'll see that PDF in the window to the right.
%
% The LaTeX macro language is complicated, so we have inserted
% lots of documenting comments into the file.  Comments start
% with `%' and continue to the end of the line.  In Overleaf's
% window, they are colored blue.
%
% Comments prefixed with `Student:' are relevant to students.
% Skip anything else you don't understand, or ask me.
%
% set font encoding for PDFLaTeX or XeLaTeX
\usepackage{ifxetex}
\ifxetex
  \usepackage{fontspec}
\else
  \usepackage[T1]{fontenc}
  \usepackage[utf8]{inputenc}
  \usepackage{lmodern}
\fi

% Student: These lines describe some document metadata.
\title{Problem Set 4}
\author{%
% Student: change the next line to your name!
    Name
\\  MATH-UA 120 Discrete Mathematics
}
\date{due March 1, 2023}


\usepackage[headings=runin-fixed-nr]{exsheets}
% These make enumerates within questions start at the second ("(a)") level, rather than the first ("1.") level.
\makeatletter
    \newcommand{\stepenumdepth}{\advance\@enumdepth\@ne}
\makeatother
\SetupExSheets{
    question/pre-body-hook=\stepenumdepth,
    solution/pre-body-hook=\stepenumdepth,
}
\DeclareInstance{exsheets-heading}{runin-nn-np}{default}{
    runin = true,
    title-post-code = .\space,
    join = {
        main[r,vc]title[l,vc](0pt,0pt);
    }
}
\newif\ifshowsolutions
% Student: replace `false' with `true' to typeset your solutions.
% Otherwise they are ignored!
\showsolutionsfalse
\ifshowsolutions
    \SetupExSheets{
        question/pre-hook=\itshape,
        solution/headings=runin-nn-np,
        solution/print=true,
        solution/name=Answer
    }%
    \makeatletter%
    \pretocmd{\@title}{Answers to }%
    \makeatother%
\else
    \SetupExSheets{solution/print=false}
\fi

% Bug workaround: http://tex.stackexchange.com/a/146536/1402
%\newenvironment{exercise}{}{}
\RenewQuSolPair{question}{solution}
%\let\answer\solution
%\let\endanswer\endsolution
\usepackage{manfnt}
\newcommand{\danger}{\marginpar[\hfill\dbend]{\dbend\hfill}}

% We are creating a command for some common commands.
\newcommand{\Z}{\mathbb{Z}}
\newcommand{\modulo}{\text{mod }}

% This package is for specifying graphics.  It's amazing.
% Manual at http://texdoc.net/texmf-dist/doc/generic/pgf/pgfmanual.pdf
\usepackage{tikz}

\usepackage{amsmath, amsthm}
\usepackage{amsfonts}
\usepackage{siunitx}
\DeclareSIUnit\pound{lb}
\usepackage{hyperref}
\newtheorem*{theorem}{Theorem}
\theoremstyle{definition}
\newtheorem*{definition}{Definition}
% This is the beginning of the part of the file that describes
% the text of the document.
% That's why it says `\begin{document}' below. :-)
\begin{document}
\maketitle



These are to be written up in \LaTeX{} and turned in to Gradescope.\\



\ifshowsolutions
    \SetupExSheets{solution/print=true}
\else
    \danger
 \underline{ \LaTeX  Instructions:}  You can view the source (\texttt{.tex}) file \href{https://bit.ly/3SX4VT4}{here} to get some more examples of \LaTeX{} code.  I have commented in the source file in places where new \LaTeX{} constructions are used.
  
  Remember to change \verb|\showsolutionsfalse| to \verb|\showsolutionstrue|
    in the document's preamble 
    (between \verb|\documentclass{article}| and \verb|\begin{document}|)
\fi

\section*{Assigned Problems}


\begin{question}{7}
    Define a relation $R$ on $\mathbb{N}\times \mathbb{N}$ by
	\[
	(a, b)\mathrel{R}(c, d) \iff 3a+d = b+3c.
	\]
	\begin{enumerate}
	\item Show that $R$ is an equivalence relation, 
	\item Describe the equivalence classes $[(0, 0)], [(1, 2)],$ and $[(2, 1)]$.
	\end{enumerate}
\end{question}
% Student: put your answer between the next two lines.
\begin{solution}
\end{solution}


\begin{question}{9}
    Are the given relations irreflexive? antisymmetric? transitive? Either \textit{prove} generally or \textit{disprove} via 
    counterexample.
    	\begin{enumerate}
	\item For $x, y \in \Z$,  $x\mathrel{R}y \iff |x - y| > 0$. 
	\item  $x\mathrel{R}y$ means that $x$ and $y$ have a common prime factor (a prime number that divides both $x$ and $y$), 
	where $x, y \in \Z$.
	\item For $x, y \in 2^{\Z}$. $x\mathrel{R}y \iff x \cap y \neq \emptyset$.
	\end{enumerate}
\end{question}
% Student: put your answer between the next two lines.
\begin{solution}
\end{solution}

\begin{question}{6}
    Determine all the (distinct) equivalence classes of the following equivalence relations. Provide a brief justification on how you found all the equivalence classes of each relation. (\textit{There is no need to prove they are equivalence relations; but you are encouraged to prove they are equivalence relations for practice.})
	\begin{enumerate}
	\item Consider the set $A = \{0, 1, 2, \dots, 8 \}$. Define a relation $R$ on $A$ by
	\[
	a\mathrel{R}b \iff a^2 \equiv b^2 \pmod{9}.
	\]
	\item Let $\mathrel{R}$ be a relation on $\mathbb{Z}$. $a\mathrel{R}b \iff 3a-5b$ is even.
	\item Let $\mathrel{R}$ be a relation on $\mathbb{Z}$. $a\mathrel{R}b \iff 4 \mid (a+3b)$.
	\end{enumerate}
\end{question}
% Student: put your answer between the next two lines.
\begin{solution}
\end{solution}

\begin{question}{4}
    Let $R$ be a relation on a set $A$. Prove $R\cup R^{-1}$ is symmetric.
\end{question}
% Student: put your answer between the next two lines.
\begin{solution}
\end{solution}

\begin{question}{18}
    \begin{enumerate}
    	\item Let $X$ be a set with $n$ elements. How many possible reflexive relations on $X$ are there?
	\item Let $A$ be a set with size $n$. For $x, y \in 2^{A}$, $x\mathrel{R}y \iff x \cap y = \emptyset$. What is the cardinality of $\mathrel{R}$?
   	\item What is the total number of partitions of $\{1, 2, \dots, 100 \}$ that has exactly two parts in the partition? 
	Remember, both parts should be non-empty.
        \item Suppose that a single character is stored in a computer using eight bits. 
        How many bit patterns have at least two 1's?
        	\item Twenty people are to be divided into two teams with ten players on each team.  
	In how many ways can this be done?
        \item Thirty five discrete math students are to be divided into seven discussion groups, each consisting of five students.  
        In how many ways can this be done?
   	\end{enumerate}
\end{question}
% Student: put your answer between the next two lines.
\begin{solution}
\end{solution}

\begin{question}{6}
    Prove \textbf{combinatorially} that
    \[ 3^n = \binom{n}{0} \cdot 2^0 + \binom{n}{1}\cdot 2^1+ \binom{n}{2}\cdot 2^2+\cdots +\binom{n}{n}\cdot 2^n. \]
    \textbf{You may not manipulate the question algebraically.}
\end{question}
% Student: put your answer between the next two lines.
\begin{solution}
\end{solution}

\end{document}
