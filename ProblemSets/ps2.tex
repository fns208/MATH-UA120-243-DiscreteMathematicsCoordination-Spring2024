\documentclass{article}
% This is a LaTeX file.  It is a text file that is compiled
% This is a LaTeX file.  It is a text file that is compiled
% by a program called LaTeX into a pretty PDF file.  
% If you're viewing this file on Overleaf, 
% you'll see that PDF in the window to the right.
%
% The LaTeX macro language is complicated, so we have inserted
% lots of documenting comments into the file.  Comments start
% with `%' and continue to the end of the line.  In Overleaf's
% window, they are colored blue.
%
% Comments prefixed with `Student:' are relevant to students.
% Skip anything else you don't understand, or ask me.
%
% set font encoding for PDFLaTeX or XeLaTeX
\usepackage{ifxetex}
\ifxetex
  \usepackage{fontspec}
\else
  \usepackage[T1]{fontenc}
  \usepackage[utf8]{inputenc}
  \usepackage{lmodern}
\fi

% Student: These lines describe some document metadata.
\title{Problem Set 2}
\author{%
% Student: change the next line to your name!
    Name
\\  MATH-UA 120 Discrete Mathematics
}
\date{due February 9, 2024}


\usepackage[headings=runin-fixed-nr]{exsheets}
% These make enumerates within questions start at the second ("(a)") level, rather than the first ("1.") level.
\makeatletter
    \newcommand{\stepenumdepth}{\advance\@enumdepth\@ne}
\makeatother
\SetupExSheets{
    question/pre-body-hook=\stepenumdepth,
    solution/pre-body-hook=\stepenumdepth,
}
\DeclareInstance{exsheets-heading}{runin-nn-np}{default}{
    runin = true,
    title-post-code = .\space,
    join = {
        main[r,vc]title[l,vc](0pt,0pt);
    }
}
\newif\ifshowsolutions
% Student: replace `false' with `true' to typeset your solutions.
% Otherwise they are ignored!
\showsolutionsfalse
\ifshowsolutions
    \SetupExSheets{
        question/pre-hook=\itshape,
        solution/headings=runin-nn-np,
        solution/print=true,
        solution/name=Answer
    }%
    \makeatletter%
    \pretocmd{\@title}{Answers to }%
    \makeatother%
\else
    \SetupExSheets{solution/print=false}
\fi

% Bug workaround: http://tex.stackexchange.com/a/146536/1402
%\newenvironment{exercise}{}{}
\RenewQuSolPair{question}{solution}
%\let\answer\solution
%\let\endanswer\endsolution
\usepackage{manfnt}
\newcommand{\danger}{\marginpar[\hfill\dbend]{\dbend\hfill}}

\usepackage{amsmath, amsthm, amssymb}
\usepackage{amsfonts}
\usepackage{enumerate}
\usepackage{siunitx}
\DeclareSIUnit\pound{lb}
\usepackage{hyperref}
\newtheorem*{theorem}{Theorem}
\newtheorem*{claim}{Claim}
\theoremstyle{definition}
\newtheorem*{definition}{Definition}
% We are creating a command "\xor".
\newcommand{\xor}{\underline{\lor}}
% This is the beginning of the part of the file that describes
% the text of the document.
% That's why it says `\begin{document}' below. :-)
\begin{document}
\maketitle



These are to be written up in \LaTeX{} and turned in to Gradescope.\\



\ifshowsolutions
    \SetupExSheets{solution/print=true}
\else
    \danger
 \underline{ \LaTeX  Instructions:}  You can view the source (\texttt{.tex}) file \href{https://bit.ly/48Wbgod}{here} to get some more examples of \LaTeX{} code.  I have commented in the source file in places where new \LaTeX{} constructions are used.
  
  Remember to change \verb|\showsolutionsfalse| to \verb|\showsolutionstrue|
    in the document's preamble 
    (between \verb|\documentclass{article}| and \verb|\begin{document}|)
\fi

\section*{Assigned Problems}


\begin{question}
    \begin{enumerate}
        \item Disprove: Every triangle has at least one obtuse angle.
        \item Disprove: For all integers $x, y$, if $x$ divides $y^2$, then $x$ divides $y$.
        \item Disprove: For every positive non-prime integer $n$, if some prime $p$ divides $n$, 
            then some other prime $q$ ($q\neq p$) also divides $n$.
        \item Disprove: For all integers $n$, if $n^5-n$ is even, then $n$ is even.
        \item Disprove: For all natural numbers $n$, the integer $n^2+17n+17$ is prime.
    \end{enumerate}
\end{question}
% Student: put your answer between the next two lines.
\begin{solution}
\end{solution}

\begin{question}
    Another method to prove that certain Boolean formulas
    are tautologies is to use the properties in Theorem~7.2
    together with the fact that $x \rightarrow y$ is equivalent
    to $(\neg x) \vee y$ (Proposition~7.3)
    For example, Exercise 7.11, part (b) asks you to establish that the formula $(x \wedge (x \rightarrow y)) \rightarrow y$ is 
    a tautology.  Here is a derivation of that fact:
    % This is an "align" environment for aligning
    % mathematical equations.  Ampersands (&) denote 
    % separation between columns.  The first one means
    % there will be alignment around the equal sign.
    % The double-ampersands put a second column, left-
    % justified.  Double-backslashes (\\) separate lines.
    \begin{align*}
        (x \wedge (x \rightarrow y)) \rightarrow y
        &= [x \wedge (\neg x \vee y)] \rightarrow y
        && \text{translate $\rightarrow$} 
        \\
        &= [(x \wedge \neg x) \vee (x \wedge y)] \rightarrow y
        && \text{distributive}
        \\
        &= [\mathrm{FALSE} \vee (x\wedge y)] \rightarrow y
        && \text{inverse elements}
        \\
        &= (x\wedge y) \rightarrow y
        && \text{identity element}
        \\
        &= \neg(x\wedge y) \vee y
        && \text{translate $\rightarrow$}
        \\
        &= (\neg x \vee \neg y) \vee y
        && \text{De~Morgan's laws}
        \\
        &= \neg x \vee (\neg y \vee y)
        && \text{associativity}
        \\
        &= \neg x \vee \mathrm{TRUE}
        && \text{inverse elements}
        \\
        &= \mathrm{TRUE}
        && \text{identity element}
        \\
    \end{align*}
    Use this technique [not truth tables] to prove that these formulas
    are tautologies:
    \begin{enumerate}
        \item $(\neg x \wedge (x \vee y)) \rightarrow y$
        \item $\neg ( (x \rightarrow y) \rightarrow  y) \rightarrow \neg x$
    \end{enumerate}
\end{question}
% Student: put your answer between the next two lines.
\begin{solution}
\end{solution}


\begin{question}
	Let the following statements be given.
		\begin{align*}
		x &= \text{``There is rain.''}\\
		y &= \text{``There is traffic.''}\\
		z &= \text{``The football game is happening.''}\\
		w &= \text{``The trophy was awarded.''}
		\end{align*}
	\begin{enumerate}
		\item Rewrite the following statements as Boolean expressions.
			\begin{enumerate}
			\item If there is no rain or no traffic, then the football game is happening.
			\item If the football game is happening, then a trophy will be awarded.
			\end{enumerate}
		\item Construct a truth table for the statements in part (a).
		\item Suppose that the statements given in part (a) are true, and suppose also that the trophy was not awarded. Did it rain? Justify your answer using the truth table.
	\end{enumerate}
\end{question}
% Student: put your answer between the next two lines.
\begin{solution}
\end{solution}
% Student: The following is an example of how to typeset a table
	% This is an "array" environment for aligning
	% individual objects. It is similar to "align"
	% Right after we \begin{array}, we indicate
	% how we want to align each individual column:
	% "c" for centered, "r" for right-justified,
	% "l" ("ell") for left-justified.
	% The total number of c's, r's, and l's should match
	% the number of columns.
	% We can use this to form a table. To create the borders,
	% we use "|" (the notation for "divides") to create a
	% vertical line between each column.
	% After each line, double-backslashes (\\) separate lines.
	% If we want a horizontal line after each line, 
	% we use "\hline".
	%For math environment, we can use "$$" or "\[ \]".
	%\[\begin{array}{| c | c || c |}
	%\hline
	%x & y & x \vee y \\
	%\hline
	%T & T & T \\ 
	%T & F & T \\
	%F & T & T \\
	%F & F & F \\
	%\hline
	%\end{array}\]

\begin{question}
    Suppose each single character stored in a computer uses eight bits. 
    Then each character is represented by a different sequence of eight 0's and 1's called a bit pattern, 
    such as 10011101 or 00100010. Provide a brief reasoning for the following questions.
        \begin{enumerate}
            \item How many different bit patterns are there?
            \item How many different bit patterns are palindromes (the same backwards as forwards)?
            \item How many different bit patterns have an even number of 1's?
            \item How many different bit patterns have the property that their second and fourth digits are 1's?
            \item How many different bit patterns have the property that their second or fourth digits are 1's?
        \end{enumerate}
\end{question}
% Student: put your answer between the next two lines.
\begin{solution}
\end{solution}

\begin{question}
    Suppose we want to make a list of length 5 from the letters $A, B, C, D, E, F, G, H, I, J$. Provide a brief reasoning for the following questions.
        \begin{enumerate}
            \item How many such lists can be made if repetition is not allowed and the list must begin with a vowel?
            \item How many such lists can be made if repetition is not allowed and the list must contain exactly one $A$?
            \item How many such lists can be made if repetition is not allowed and the list must begin with a vowel and end with a vowel?
            \item How many such lists can be made if repetition is not allowed, and the list must begin with a vowel or end with a vowel?
            \item How many such lists can be made if repetition is not allowed and the list must contain exactly two vowels?
        \end{enumerate}
\end{question}
% Student: put your answer between the next two lines.
\begin{solution}
\end{solution}

\end{document}
