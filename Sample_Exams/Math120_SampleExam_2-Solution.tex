\documentclass[10pt, addpoints]{exam}
\usepackage[margin=1in]{geometry}
\usepackage{amsmath, amssymb, amsthm, graphicx, hyperref}
\usepackage{enumerate}
\usepackage{color}
\usepackage{graphicx}
\usepackage{epstopdf}
\DeclareGraphicsRule{.tif}{png}{.png}{`convert #1 `dirname #1`/`basename #1 .tif`.png}
\usepackage{multirow, multicol}
\usepackage{pdfpages}
\usepackage{pgfplots}
\usepackage{wrapfig}
\usepackage{stfloats}
\usepackage{tikz}
\usetikzlibrary{arrows.meta}
\usepackage{wasysym}
\firstpageheader{}{}{}
\firstpagefooter{}{}{}
\runningheader{\scalebox{0.25}{\includegraphics{nyu_short_black}}}{}{\tiny Math-UA.120}
\runningfooter{}{\thepage}{}

\newtheorem*{theorem}{Theorem}


\printanswers


%% MC Solution Sheet ----------------------------
\newcommand*\mycirc[1]{%
  \begin{tikzpicture}[baseline=(C.base)]
    \node[draw,circle,inner sep=2pt](C) {#1};
  \end{tikzpicture}}
\renewcommand\choicelabel{\mycirc{\Alph{choice}}}
%% ----------------------------------------------
  
\theoremstyle{definition}
\newtheorem*{definition}{Definition}

\newcommand{\R}{\mathbb{R}}
\newcommand{\E}{\mathbb{E}}
\newcommand{\Z}{\mathbb{Z}}
\newcommand{\N}{\mathbb{N}}
\newcommand{\Q}{\mathbb{Q}}
\newcommand{\proj}{\textbf{proj}}
\newcommand{\curl}{\textrm{curl }}
\renewcommand{\div}{\textrm{ div }}
% For boldface vectors:
\renewcommand{\vec}[1]{\mathbf{#1}}
\renewcommand\arraystretch{2}
 
\newcommand*\ansbox[2]{\fbox{\begin{minipage}{#1}
     \hspace{#1}
     \vspace{#2}
     \end{minipage}}} 
%%%Preamble for Header
\pagestyle{headandfoot}
\firstpageheadrule
\runningheadrule
\firstpageheader{\includegraphics[height=.5cm]{nyu_short_black}}{Exam 2, Page \thepage\ of \numpages}{\tiny MATH-UA.120 }
\runningheader{\includegraphics[height=.5cm]{nyu_short_black}}
              {Exam 2, Page \thepage\ of \numpages}
              {\tiny MATH-UA.120 }
\firstpagefooter{}{}{}
\runningfooter{}{}{}
%\newcommand*\mycirc[1]{%
%  \begin{tikzpicture}[baseline=(C.base)]
%    \node[draw,circle,inner sep=1pt](C) {#1};
%  \end{tikzpicture}}
\renewcommand\arraystretch{2}
 


\pgfplotsset{compat=1.16}

\begin{document}

\addpoints

\extrawidth{.25in}
%\pagestyle{plain}
\usehorizontalhalf
\setlength\topmargin{-.75in}


%%%%%%%%%%%%%%%BEGINNING-OF-TITLEPAGE%%%%%%%%%%%%%%%%%
\begin{titlepage}
\noindent{\large\textbf{Math-UA.120}  \hfill \includegraphics[height=.5cm]{nyu_short_black}}\\
\noindent{
\textbf{Sample Exam 2} 
}


\noindent \begin{center}\begin{tabular}{|cccc|}\hline
\large Name: & \hspace{4in} 
&
\large NetID: & \hspace{1in} \\
&&&\\
%NetID:& \\
 \hline
 \end{tabular}\end{center}
\vspace{.3in}
\noindent\textbf{Please read and follow directions carefully}: 

\begin{itemize}
	\setlength\itemsep{0em}
	\item Please \textbf{clearly print} your full name and NetID (not your N number).
	\item This exam is scheduled for 100 minutes.
\item This exam is printed double-sided.
\item A blank page has been attached to the exam; it may be used as scratch paper.
\item Do \textbf{NOT} detach any page from the exam, except scratch paper.
\item During this exam, you may only use the scratch paper provided and writing utensils. \textbf{No calculators}, cell phones, books, notes or other resources will be permitted.
	\item There are $\numquestions$ total questions on this exam for a total of $\numpoints$ points.
	\item There are 3 types of questions: single-choice, multiple-choice, and free-response. 
	\begin{itemize}
	\item \textbf{Single and Multiple-choice} requires no justification. 
	\item \textbf{Free-response} questions should have solutions that are fully justified in the space provided.
	You must present a full solution in a clear, logical manner that conveys your full understanding of the concept being tested. Correct answer without justification \underline{receives no credit.}
	\item We reserve the right to take off points if we cannot see how you arrived at your answer (even if your final answer is correct).
	\end{itemize}

%\item Clearly circle/box your final answers, with the exception of proofs.

%\item Be sure to erase or cross out any work that you do not want graded.


\end{itemize}

\vspace{0.3in}


\vfill
\begin{center}
\emph{By sitting for this exam, I pledge that I have observed the NYU honor code, and that I have neither given nor
received unauthorized assistance during this examination.}
\end{center}


\begin{center}
 {\large \bf GOOD LUCK :)}
\end{center}

\end{titlepage}

\pointsinrightmargin
%%%%%%%%%%%%%%%%END-COVERSHEET%%%%%%%%%%%%%%%%%
\renewcommand\arraystretch{1}

\newpage
%%%%%%%%%%%%%%%%%%%%MultipleChoice%%%%%%%%%%%%%%%%
\begin{questions}
\section*{Multiple Choice: Single Correct Answer} 

\question For each multiple choice question, \textbf{clearly bubble your choice}. There is only one correct choice. No justification is required. No partial credit will be given.
\begin{parts}
\part[2] Suppose $f: \Z \to \Z$ is given by the formula $f(x) = 2x$.

Which one of the following is the domain of $f^{-1}$ ? 
\begin{choices}
\choice $\{ \frac{x}{2} ~:~ x\in \Z\}$\\
\CorrectChoice $\{ 2x ~:~ x\in \Z\}$\\ %T
\choice $\Z$\\
\choice $\N$\\
\choice None of the above
\end{choices}

\vfill
Suppose that $A = \{ S \subseteq \{1, 2, \ldots, m \} ~:~ |S| = 2 \}$ and 
$B = \{1, 2, 3, \ldots, n \}$, where $m$ and $n$ are positive integers. Consider an arbitrary function $f: A \to B$. 

Which one of the following statements is \textbf{true}?

\begin{choices}
\choice If $m < n$, then $f$ one-to-one.\\
\choice If $f$ is one-to-one, then $m \leq n$.\\
\CorrectChoice If ${m \choose 2} > n$, then $f$ not one-to-one.\\%T
\choice If $f$ is not one-to-one, then ${m \choose 2} \geq n$.\\
\choice None of the above
\end{choices}

\vfill
\newpage
\part[2] If a Martian has an infinite number of blue, yellow, and black socks in a drawer, how many socks does the Martian need to pull out from the drawer to guarantee he has a pair of the same color?

\begin{choices}
\choice $2$\\
\choice $3$ \\
\choice $4$\\
\CorrectChoice $5$\\%T
\choice None of the above
\end{choices}

\vfill
\part[2] Let $A=\{1, 2, 3\}$ and $B=\{3, 4, 5, 6\}$ be sets. How many functions are there from $A$ to $B$?

\begin{choices}
\choice ${4\choose3}$ \\
\choice $3$\\
\choice $(4)_3$ \\
\CorrectChoice $4^3$ \\ %T
\choice None of the above
\end{choices}

\vfill
\part[2] Here is a flawed proof that all natural numbers are equal to $0$:
    \begin{enumerate}
        \item Let $A = \{0\}$.  We claim $A = \mathbb{N}$.
        \item Clearly $0\in A$.
        \item Suppose $0,1,\dots,k \in A$ for some $k\in\mathbb{N}$.  Then $1 \in A$ and $k\in A$.
        \item So $k+1 = 0 + 0 =0$, which means $k+1\in A$.
        \item Therefore by strong induction, $A = \mathbb{N}$.
    \end{enumerate}
 In which line is the error?
\begin{choices}
\CorrectChoice {line 3}\\%correct
\choice {line 4}\\
\choice {line 2}\\
\choice{line 5}\\
\choice{line 1}
    \end{choices}

\vfill

\end{parts}
\newpage

\section*{Multiple Choice: Multiple Correct Answers} 

\question For each multiple choice question, \textbf{clearly bubble your choice(s)}. There may be more than one correct choice; \textbf{fill in all that apply}. If none of the choices apply, fill in "None of the above". No justification is required. Partial credit will be given.

\begin{parts}
%%Cardinality
\part[4] Which of the following sets have the same cardinality as the set of natural numbers?

\begin{choices}
\CorrectChoice The set of rational numbers that are between 0 and 1.\\
\CorrectChoice The set of all even integers.\\
\choice The power set of the set $S = \{1, 2, 3, 4\}$\\
\CorrectChoice The set of all integers that are divisible by 5.\\
 \choice None of the above
\end{choices}
%\input{MC-Multi-PHP-sock}
%%
\vfill
%%
%\newpage
\part[4] Which of the functions below are one-to-one ?

\begin{choices}
\CorrectChoice $f: \Q \to \Q$ given by $f(x) = 7 - 2x$\\
\CorrectChoice $g: \Z \to \Z$ given by $g(x) = 7 - 2x$\\
\CorrectChoice $h: \N \to \Z$ given by $h(x) = |x|$\\
\choice $k: \Z \to \Z$ given by $k(x) = x^2 + 5$\\%F
\choice None of the above
\end{choices}
\vfill

%\newpage
\part[4] Which of the functions below are onto? 

\begin{choices}
\CorrectChoice $f: \Q \to \Q$ given by $f(x) = 7 - 2x$\\%onto
\choice $g: \Z \to \Z$ given by $g(x) = 7 - 2x$\\
\choice $h: \N \to \Z$ given by $h(x) = |x|$\\
\choice $k: \Z \to \Z$ given by $k(x) = x^2 + 5$\\
\choice None of the above
\end{choices}
\vfill

%%\newpage
%\input{MC-Multi-Sets}
%\vfill
%%\input{MC-SubGraph}
%\vfill


%\vfill
\end{parts}

%%%%%%%%%%%FREE RESPONSE%%%%%%%%%%%%

\newpage
\section*{Free Response/Proofs}
\noindent
\textbf{Show all work and justification.} Correct answer without justification receives no credit.
%%%%%%%%%%%%%%%%%%%%%%%%%%%%%%%
\question
\begin{parts}
\part[5] Suppose we sample $W$ words at random from an english language dictionary. What is the smallest integer $W$ such that at least five of the sampled words start and end with the same letters? Use the pigeonhole principle to support your answer.
\begin{solution}
Since there are 26 letters in the alphabet, there are $26^2$ possible pairs of first and last letters. According to the generalized pigeonhole principle, we need to assume that for every pair of first and last letters we have 4 copies and then add one more word to ensure it will be the fifth copy of one of the pairs of first and last letters. That is, we need $4(26^2)+1$ words to guarantee that at least five of the sampled words start and end with the same letters.
\end{solution}
\vfill

\part[5] Use the pigeonhole principle to find the smallest number of ways to choose three different numbers from 1 to 10 (inclusive) so that all choices have the same sum.
\begin{solution}
Choosing from $\{1, 2, \dots, 10\}$, there are $\binom{10}{3}=120$ ways to choose three different numbers, which are the pigeons. The smallest possible sum from any three numbers is $1+2+3 = 6$ and the largest sum from any three numbers is $8+9+10=27$. Then we have a sum that can range from 6 to 27; that is we have 22 different possible sums, which are the pigeonholes. Since $120 > 5(22) = 110$, by the generalized  pigeonhole principle, there are at least 6 sets of three numbers choosing from $\{1, 2, \dots, 10\}$ that have the same sum.
\end{solution}
\vfill

\end{parts}

\newpage
% Easy induction
\question[8] Prove the following statement using \textbf{smallest counterexample}.
\begin{quotation}
For each $n \in \N, \ n \geq 1$:
\[ 5 + 8 + 11 + 14 + \ldots + (3n+2) = \frac{n(3n+7)}{2}. \]

\end{quotation}

\begin{solution}
For the sake of contradiction, assume the statement is not true.

Let $X=\{n\in \N : n\geq 1, 5 + 8 + 11 + 14 + \ldots + (3n+2) \neq \frac{n(3n+7)}{2}\}.$ Then $X\neq \emptyset$. By the Well-Ordering Principle, there exists $x\in X$ that is the least element. Note $x\neq 1$ because $3(1)+2 = 5$ and $\frac{1(3(1)+7)}{2} = 5$. Then $x>1$. Consider $x-1$. Since $1\leq x-1 <x$, 
\[ 5 + 8 + 11 + 14 + \ldots + (3(x-1)+2) = \frac{(x-1)(3(x-1)+7)}{2}. 
\]
Observe that 
\begin{align*}
 5 + 8 + 11 + 14 + \ldots + (3(x-1)+2)  + (3x+2) & = \frac{(x-1)(3(x-1)+7)}{2} + (3x+2)\\
 & = \frac{(x-1)(3x+4)}{2} + 3x+2 \\
 & = \frac{3x^2+x-4+6x+4}{2}\\ 
 &= \frac{3x^2+7x}{2}\\
 &= \frac{x(3x+7)}{2}.
\end{align*}
Contradiction! Therefore the statement is true.
\end{solution}


\newpage
\question Given $f:A\to B$, let $g:2^B\to 2^A$ be defined as follows: for any $X\in 2^B$, 
\[ g(X)=\{a\in A: f(a)\in X\}.\]
\begin{parts}
\part[4] Suppose, for example, $A=\{1, 2, 3, 4\}$, $B=\{5, 6, 7\}$, and $f=\{(1, 5), (2, 7), (3, 5), (4, 6)\}$. 
Find $g(\{5\})$ and $g(\{6, 7\})$. 
\begin{solution}
$g(\{5\}) = \{1, 3\}, g(\{6, 7\}) = \{2, 4\}$
\end{solution}
\vfill

\part[10] Prove and/or disprove: If $g$ is onto, then $f$ is one-to-one and onto.
\begin{solution}
\begin{itemize}
\item[onto:] We will disprove $f$ is onto by counterexample. Suppose $A=\{1\}, B=\{2, 3\},$

$ 2^A=\{\emptyset, \{1\}\}, 2^B = \{\emptyset, \{2\}, \{3\}, \{2, 3\}\},$ and $g= \{(\emptyset, \emptyset), (\{2\}, \{1\}), (\{3\}, \emptyset), (\{2, 3\}, \{1\})\}$. Then $g$ is onto. Note that $f=\{(1, 2)\}$ which is not onto. Therefore the statement `` If $g$ is onto, then $f$ is onto.'' is false.

\item[1-1:] Assume $g$ is onto. For the sake of contradiction, assume $f$ is not one-to-one. Let $a_1, a_2\in A, b\in B$ such that $f(a_1)=f(a_2)=b$ and $a_1\neq a_2$. Then $g(\{b\})=\{a_1, a_2\}$. Since $g$ is onto, there exist $b_1, b_2\in B$ such that $g(\{b_1\})=\{a_1\}$ and $g(\{b_2\})=\{a_2\}$. Then $f(a_1)=b_1$ and $f(a_2)=b_2$. Since $f$ is a function, $b_1=b=b_2$, which implies $\{b_1\}=\{b_2\}$. Since $g$ is a function, $\{a_1\}=\{a_2\}$, which implies $a_1=a_2$. Contradiction! Therefore, the statement `` If $g$ is onto, then $f$ is one-to-one.'' is true.
\end{itemize}
\end{solution}
\vfill
\vfill
\vfill
\vfill
\end{parts}

\newpage
%%Contrapositive - three cases
\question[8] Prove by contrapositive: Suppose $x,y\in \Z$. If both $xy$ and $x+y$ are even, then both $x$ and $y$ are even.

\begin{solution}
	Suppose not both $x$ and $y$ are even. Then at least one of them is odd. We want to show that not both $xy$ and $x+y$ are even. 
	\begin{itemize}
	\item[Case 1:] Suppose $x$ is even and $y$ is odd. Then there exist $a,b\in \Z$ such that $x=2a$ and $y=2b+1$. Observe $xy=(2a)(2b+1) = 2(a(2b+1))$ is even and $x+y = 2a+2b+1 = 2(a+b)+1$ is odd. Thus not both $xy$ and $x+y$ are even.
	\item[Case 2:] Suppose $x$ is odd and $y$ is even. Then there exist $a, b\in \Z$ such that $x=2a+1$ and $y=2b$. Observe $xy = (2a+1)(2b)= 2(b(2a+1))$ is even and $x+y = (2a+1) + 2b = 2(a+b)+1$ is odd. Thus not both $xy$ and $x+y$ are even.
	\item[Case 3:] Suppose $x$ is odd and $y$ is odd. Then there exist $a, b\in \Z$ such that $x=2a+1$ and $y=2b+1$. Observe $xy = (2a+1)(2b+1)= 4ab+2a+2b+1 = 2(2ab+a+b)+1$ is odd and $x+y = (2a+1) + (2b+1) = 2(a+b+1)$ is even. Thus not both $xy$ and $x+y$ are even.
	\end{itemize}
	Therefore, if not both $x$ and $y$ are even, then not both $xy$ and $x+y$ are even. By the contrapositive, if both $xy$ and $x+y$ are even, then both $x$ and $y$ are even.
\end{solution}
\newpage
\question[10] For each $n\in \N$, let $H_n$ be defined as follows: 
$H_0=3, \quad H_1=8,$ for all $n\in \N$ where $n\geq 2$:
\[ H_n=4H_{n-1}-4H_{n-2}.\]

Prove for all $n\in \N$, $\displaystyle H_n=(3+n)2^n$.
\textit{Hint: Use Strong Induction.}

\begin{solution}
\begin{description}
\item[Base Case:] Consider $n=0$: $H_0 = 3 = (3+0)2^0$. Consider $n=1$: $H_1 = 8 = (3+1) 2^1$.
\item[Inductive Hypothesis:] Let $k\geq 1$ be an arbitrary natural number such that $0\leq n\leq k$. Assume $H_n = (3+n)2^n$.
\item[Inductive Step:] Consider $n=k+1$. We want to show $H_{k+1} = (3+(k+1))2^{k+1}$. Observe that 
\begin{align*}
H_{k+1} & = 4 H_k - 4H_{k-1}\\
&= 4(3+k)2^k -4(3+(k-1))2^{k-1} \quad \text{(by Inductive Hypothesis)}\\
&= 4\cdot 2^{k-1}((3+k)2 - (2+k))\\ 
&= 2^{k+1} (k+4)\\
& = 2^{k+1} ( 3 + (k+1)).
\end{align*}
\end{description}
Therefore, strong induction, for all $n\in \N$, $\displaystyle H_n=(3+n)2^n$.
\end{solution}
\end{questions}

%%%%%%%%%%%%%%%END-OF-FREE-RESPONSE%%%%%%%%%%%%%%%%%



%%%%%%%%%%%%%%%SCRAP%%%%%%%%%%%%%%%%%
%\newpage
%%Scrap
%\section*{Extra Paper}
%
\newpage
%\pagestyle{empty}
\section*{This page is intentionally left blank for computations!!!}

%\newpage
%\section*{This page is intentionally left blank for computations!!!}
%\newpage
%%\pagestyle{empty}
%\section*{This page is intentionally left blank for computations!!!}
%
%\newpage
%\section*{This page is intentionally left blank for computations!!!}
%%%%%%%%%%%%%%%SCRAP%%%%%%%%%%%%%%%%%

\end{document}  













